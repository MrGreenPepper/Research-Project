\documentclass[british,         % thesis language
BCOR=2mm,                       % binding correction for printing (ca. 1mm per 10 pages for hardcover binding)
11pt,                           % font size
a4paper,						% page layout
oneside,						% oneside/twoside option for printing
cdgeometry,                     % setting of page geometry (format, height, page margins; set to 'cdgeometry=false' for individual type area calculation by package 'typearea'
toc=chapterentrydotfill,        % dots in table of contents
toc=indent,                     % indentation of toc items
bibliography=totoc,         	% bibliography to table of contents
listof=totoc,                   % lists to table of contents
numbers=noenddot,				% last number in table of contents without dot
parskip=full,                   % line break with a free line
cdmath=false					% regular font in math mode
]{article}                  % document class ('tudscrreprt' in most cases)


%\usepackage{titlesec}		
\usepackage{scrextend}			% für addmargin zum einrücken von Absätzen   \begin{addmargin}[25pt]{0pt} \end{addmargin}

\setlength{\headheight}{40pt} 

\title{Exposee Modellgestützte Analyse von Vermarktungsoptionen für Batteriespeicher gekoppelte Wind- und PV-Systeme}
\author{Sebastian Trümper}
\date{ }

\begin{document}
%\headlogo[]{bilder/ee_logo.png}

\maketitle

\tableofcontents


\section{Einleitung}
Aufgrund neuer gesetzlicher Vorgaben ist der Regelleistungsmarkt, 
insbesondere der Sekundärregelleistungsmarkt, 
für Anbieter erneuerbarer Energien interessanter geworden. Ein besonderes Augenmerk liegt dabei auf die mögliche Erweiterung des Energieparks 
um eine Speichermöglichkeit. Diese Arbeit soll dazu dienen den grundsätzlichen Wert einer solchen Speichermöglichkeit
am Sekundärregelleistungsmarkt zu ermitteln.
Im Weiteren ist fraglich, ob ein zusätzlicher Nutzen des Stromspeichers durch direkte Kombination mit einem 
Wind-/Solarpark erzielt werden kann.


\section{Genaue Ausgestaltung der Forschungsfrage}
Gegenstand dieser Arbeit ist die Wertermittlung einer solchen Speichermöglichkeit innerhalb des deutschen sekundären Regelleistungsmarktes. 
Dabei gilt ein besonderes Augenmerk den Unsicherheiten hinsichtlich des Preises und des Wetters bzw. der Wettervorhersagen.


\section{Forschungsvorhaben}
Um die wissenschaftliche Frage zu beantworten, soll ein Modell zur Analyse erstellt werden. 
Das Modell soll folgende Bereiche abdecken: 
\begin{itemize}
	\item Modell eines Stromspeichers (Kapazität von 2h als Industriestandard)
	\item Modell des Regelleistungsmarktes insbesondere des sekundären Regelleistungsmarktes
	\item Abbildung der Unsicherheiten hinsichtlich Wetter, Strom- und Regelpreis
\end{itemize}



\section{Methodik}
(Es gibt eine Variablenzusammenfassung am Ende dieses Abschnitts)\\
Im Kontext der Speicherwertermittlung sind mehrere Szenarien zu betrachten, 
diese werden die Unsicherheiten hinsichtlich des Preises und des Wetters abbilden.

Hierzu sind mögliche Korrelationen zwischen dem Preis und der Wettervorhersage zu beachten.
Um die Korrelation zu analysieren, werden die Preise mit 
der Wettervorhersage kombiniert und in einem 2D-Diagramm abgebildet. Anhand dieses 2D-Diagramms wird
dann mittels der "Least-Square-Methode" die Korrelation berechnet. Um einen eventuellen Einfluss der Jahreszeiten 
auf diese Korrelation zu berücksichtigen, wird hierzu ein "Mixed-Effect-Model" zusätzlich angewendet.\\

Zur Analyse des Wertes des Stromspeichers wird ein 2-stufiges stochastische Entscheidungsmodell etabliert. 
In der ersten Stufe werden Gebote für den Regelleistungsmarkt abgegeben. In der zweiten Stufe werden Gebote für den Regelarbeitsmarkt
abgegeben. Zu beachten ist das beide Märkte "pay as you bid" Märkte sind, die Höhe des Gewinns ist also direkt vom 
gebotenen Preis abhängig.\\

%Eventuell den Grad der Unsicherheit der Wettervorhersage mit rein nehmen%

(das Folgende ist jeweils für neg. und positive Regelleistungsmarkt aus zu fertigen)


\begin{flushleft}\textbf{Zu optimierende math. Terme:}\end{flushleft}
\begin{tabular}{l l}
	$\chi_s * \lambda_s^l * P(\lambda_s^l, PBV)$& 		mögl. Gewinn Regelleistungsmarkt\\
	$\gamma_s * \lambda_s^a * P(s,\lambda_s^a)$&		mögl. Gewinn Regelarbeitsmarkt\\
\end{tabular}



\begin{flushleft}\textbf{Optimierungsfunktion:}\\\end{flushleft}
\begin{addmargin}[25pt]{0pt}
$\max \pi = \chi_s * \lambda_s^l * P(\lambda_s^l, PBV) + \sum_{s} \gamma_s * \lambda_s^a *P(s,\lambda_s^a)$\\
\end{addmargin}

\begin{flushleft}\textbf{Nebenbedingungen:}\end{flushleft}
\begin{addmargin}[25pt]{0pt}
	$ \gamma_s \geq \chi_s$ \\%Die Gebotsmenge am Regelarbeitsmarkt muss min. dem der am Regelleistungsmarkt entsprechen\\
	$ \gamma_s, \chi_s \leq K$\\ % Gebotsmenge darf Netzanschluss nicht übersteigen
	$\chi_s = \chi_{s+1}$\\
	$\lambda_s^l = \lambda_{s+1}^l $ \\
\end{addmargin}



Bei einer möglichen Modellerweiterung um einen Wind-/Solarpark würde sich dann die Optimierungsfunktion
um einen möglichen Gewinn durch den Park erweitern.\\


	\begin{itemize}
		\item $ P(s, \lambda_s^a) * \psi_s * \lambda_s^e$ mögl. Einspeisung im Falle das \\Regelarbeitsmarkt-Gebot Zuschlag erhält\\
		\item $ (1 - P(s, \lambda_s^a)) * \psi_s * \lambda_s^e$ mögl. Einspeisung im Falle das \\Regelarbeitsmarkt-Gebot nicht den Zuschlag erhält\\
		\item 	$c(\gamma_s, \psi_s) = \left\{
			\begin{array}{ll}
				0 & \gamma_s + \psi_s \leq K \\
				C * (\gamma_s + \psi_s - K) & \gamma_s + \psi_s \geq K \\
			\end{array}
		\right. $ Kosten/Strafe für Nichterbringung der Leistung\\

	\end{itemize}
	
	


Damit würde sich dann die folgende, erweiterte Optimierungsfunktion ergeben:\\
\begin{addmargin}[25pt]{0pt}
	$\max \pi = \chi_s * \lambda_s^l * P(\lambda_s^l, PBV) \\
	+ \sum_{s} \gamma_s * \lambda_s^a * P(s,\lambda_s^a)\\
	+P(s, \lambda_s^a) * \psi_s * \lambda_s^e\\
	+(1 - P(s, \lambda_s^a)) * \psi_s * \lambda_s^e\\
	- \left\{
		\begin{array}{ll}
			0 & \gamma_s + \psi_s \leq K \\
			C * (\gamma_s + \psi_s - K) & \gamma_s + \psi_s \geq K \\
		\end{array}
	\right. $
	\end{addmargin}


\begin{tabular}{ c l }
	\textbf{Variablen:}&\\
	$\pi$& Gewinn \\
	$ \rho $& Prognose \\
	$ B  $& Bedarf \\
	$ PBV = \rho / B$ & Prognose-Bedarf-Verhältnis\\
	$ s $& Szenario \\
	$ \chi_s $ & Regelleistungsmarkt Gebots-Menge \\
	$\lambda_s^l$ & Regelleistungsmarkt Gebots-Preis \\
	$ P(\lambda_s^l, PBV)$& Wahrscheinlichkeit für Zuschlag bei Preis $\lambda_s^l$ am Regelleistungsmarkt\\
	$ \gamma_s  $ & Regelarbeitsmarkt Gebots-Menge \\
	$ \lambda_s^a  $& Regelarbeitsmarkt Gebots-Preis \\
	$ P(s, \lambda_s^a)$& Wahrscheinlichkeit für Zuschlag bei Preis $\lambda_s^a$ am Regelarbeitsmarkt\\
	$ P(s) $& Wahrscheinlichkeit für Szenario s \\
	$ K $& Kapazität Netzanschlusspunkt\\
	$ \lambda_s^e $& Strompreis für reguläre Einspeisung \\
	$ \psi $ & Strommenge der regulären Einspeisung \\
	$ c(\gamma_s, \psi)$& Kosten für Strafe \\
	$ C $& Strafe für nicht erbrachte Leistung \\
\end{tabular}\\


\section{Ablaufplan}

\subsection{Basismodell}
\begin{enumerate}
	\item Modellierung des Stromspeichers inkl. Netzanschlusspunkt
	\item Modellierung des Regelleistungsmarktes
	\item Modell für Vermarktung des Stromspeichers am Regelleistungsmarkt
	\item Testdaten für eine Simulation von Punkt 1.–3. in das Modell einspeisen
	\item Kontrollieren, ob das Modell realistische Werte erzeugt
	\item Ladesteuerung des Stromspeichers für den Regelarbeitsmarkt
	\item Korrelation zwischen den verschiedenen Szenarion untersuchen
	\item Formeln und Daten anhand von 7. aufarbeiten und einspeisen
	\item Optimierung und Verfeinerung des Modells (z.B. diverse Speichertechnologien mit detaillierten Lastverläufen)
\end{enumerate}

\subsection{Modell Erweiterung}
\begin{enumerate}
	\item 	Modellierung von Wind-/Solarpark
	\item 	Modellierung des primären Strommarktes
	\item 	Vermarktung des Wind-/Solarpark Stroms am primären Strommarkt
	\item 	Kombination der Basis sowie der Modellerweiterung
\end{enumerate}

Optional: Speicheraufteilung zu verschiedenen Zwecken

\subsection{Wissenschaftliche Dokumentation der Arbeit}
\begin{enumerate}
	\item Fragestellung und Relevanz aufzeigen
	\item 	Methodik Erklärung
	\item 	Modell Erklärung
	\item 	Daten Erklärung
	\item 	Auswertung der Ergebnisse
	\item 	Ergebnisse in Kontext setzten
	\item 	Zusammenfassung schreiben
	\item 	Schlussfolgerungen ziehen
\end{enumerate}

\end{document}