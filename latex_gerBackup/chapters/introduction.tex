\chapter{Introduction}

1. Einführung ins Thema
Kurze Darstellung des Themas.

Warum ist das Thema relevant (wissenschaftlich, gesellschaftlich, praktisch)?

Aktueller Forschungsstand oder gesellschaftlicher Kontext (je nach Fachgebiet).

Beispiel:
„In den letzten Jahren ist das Interesse an nachhaltiger Stadtentwicklung stark gestiegen. Besonders die Rolle grüner Infrastruktur wird dabei zunehmend als zentral betrachtet.“

2. Problemstellung
Was genau ist die Forschungsfrage oder das Problem?

Gibt es eine Forschungslücke oder ein konkretes Problem, das du adressierst?

Beispiel:
„Trotz umfangreicher Forschung zur Wirkung urbaner Grünflächen fehlen Studien zur langfristigen Wirkung auf die mentale Gesundheit in dicht besiedelten Quartieren.“

3. Zielsetzung und Forschungsfrage(n)
Was willst du mit der Arbeit erreichen?

Welche Forschungsfrage(n) leitest du daraus ab?


Beispiel:
„Ziel dieser Arbeit ist es, die Wirkung von urbaner Begrünung auf das subjektive Wohlbefinden von Stadtbewohner:innen zu untersuchen. Die zentrale Forschungsfrage lautet daher: Welche Effekte hat grüne Infrastruktur auf das psychische Wohlbefinden in urbanen Räumen?“

4. Methodisches Vorgehen (kurz)
Wie wirst du vorgehen (z.B. Literaturarbeit, empirisch, qualitativ/quantitativ)?

Beispiel:
„Zur Beantwortung der Forschungsfrage wird eine qualitative Inhaltsanalyse von Experteninterviews durchgeführt.“

5. Aufbau der Arbeit
Wie ist die Arbeit strukturiert?

Kurzer Überblick über die Kapitel.

Beispiel:
„Kapitel 2 stellt die theoretischen Grundlagen dar, Kapitel 3 erläutert das methodische Vorgehen, Kapitel 4 präsentiert
die Ergebnisse und Kapitel 5 diskutiert diese im Kontext der Forschungsfrage.“\\
----------------------------------------\\
-- positive synergie plant/battery

Im Allgemeinen ok, ich lasse das erstmal so stehen und überarbeite dann wenn ich genauer weiß wo genau der rote Pfaden liegen wird.\\


The accelerating transition towards renewable energy sources presents both opportunities and challenges for modern
power systems. But, the inherent variability and limited predictability of renewable energy generation pose
challenges to grid stability and economic efficiency. As a result, flexible technologies such as battery energy
storage systems (BESS) have become increasingly important to ensure reliable and market-efficient integration of
renewable resources.

When deployed in conjunction with renewable energy plants, battery storage systems offer complementary capabilities.
While wind farms primarily participate in the day-ahead market based on forecasted production,
battery systems can operate more strategically by responding rapidly to price signals and grid requirements.

\todo{hier noch mit rein das batterie speicher in kombination mit einem windpark positive spillovers haben kann}

In particular, the provision of ancillary services—especially frequency regulation has emerged as a promising
revenue stream for storage technologies. Germany's balancing markets, including the secondary
control reserve (aFRR), offer significant potential for batteries due to their fast ramping capabilities and
high availability.

Combining renewable energy generation with battery storage in a co-located hybrid system allows operators to diversify
revenue streams by participating simultaneously in multiple electricity markets.
\todo{- vor und nachteile, begrenzte kapazität des anschlusspunkts, überschüssige energie kann zum laden genutzt werden.}
However, such joint operation
requires sophisticated optimization techniques that consider market mechanisms, physical constraints,
and operational synergies. In this context, mathematical programming tools such as GAMS (General Algebraic Modeling System)
are well-suited to model and solve complex multi-market dispatch problems.

This paper presents a mixed-integer optimization model developed in GAMS to simulate the joint operation of a
wind farm and a co-located battery storage system. The wind farm's revenues are derived from the German day-ahead
electricity market, while the battery system participates in the secondary balancing market.
The model aims to maximize total system profit while adhering to market and technical constraints.

(This work contributes to the growing body of research on hybrid renewable systems by addressing the
integration of distinct market participation strategies and quantifying the economic benefits of coordinated operation.)
-----------------------------------------\\

\todo{forschungsfrage ergänzen}