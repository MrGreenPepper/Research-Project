\chapter{Conclusion}

Die vorliegenden Analysen zeigen deutlich, dass eine zunehmende Marktdurchdringung durch volatile Energieerzeuger
signifikante Auswirkungen auf die Inanspruchnahme sowie die Preisbildung von Regelarbeit hat.
Insbesondere ist zu beobachten, dass mit steigendem Anteil volatiler Erzeugung die Menge aktivierter negativer Regelarbeit zunimmt,
während gleichzeitig die Nachfrage nach positiver Regelarbeit zurückgeht. Dieses Muster spiegelt sich auch in den Grenzpreisen wider:
Während die Preise für negative Regelarbeit in Abhängigkeit zur Volatilität der Erzeugung steigen, sinken die Preise für positive
Regelarbeit tendenziell.

Auffällig sind zudem starke Preisausreißer bei der positiven Regelarbeit in Szenarien mit geringer oder mittlerer Marktdurchdringung
durch volatile Erzeuger. Diese scheinen auf unerwartet hohe Nachfragespitzen zurückzuführen zu sein, welche in Szenarien auftreten,
in denen Marktteilnehmer nicht mit großen Schwankungen gerechnet haben. Daraus lässt sich schließen, dass die Erwartungshaltung und
Vorbereitung der Marktakteure auf volatile Einspeisung entscheidend zur Preisstabilität beiträgt.

Die analysierten Kapazitätspreise zeigen ein differenziertes Bild: Die Medianwerte peaken sowohl für positive als auch negative
Regelarbeit im mittleren Szenario. Dies deutet auf eine erhöhte Wettbewerbsintensität in diesen Szenarien hin. Gleichzeitig
lässt sich aus den unteren Quantilen ableiten, dass bei negativer Regelarbeit selbst im Vergleich von Q1 zu Q36 kaum Unterschiede
bestehen, während bei positiver Regelarbeit gegen Tagesende eine stärkere Preisdivergenz sichtbar wird. Daraus kann geschlossen werden,
dass Anbieter in Szenarien mit erwarteter hoher volatilät für den Folgetag mit einer Regelarbeitsbereitstellung rechnen und dabei den Kapazitätspreise
als opportunistischen  „Mitnahmepreis“ gestalten. Dies führt dazu, dass das Angebot in Relation zur Nachfrage überproportional
steigt und folglich sinkende Arbeitspreise resultieren.

Die Gebotsstrategien für den Regelleistungsmarkt bewegen sich über alle Szenarien hinweg knapp unterhalb des erwarteten Grenzpreises.

In den Szenarien mit niedriger und mittlere volatiler Producktion zeigt eine verschiebung hin zum negativen Regelleistungsmarkt. Dies zeigt sich
in einem regelmäßigeren Bereitstellung von negativer Regelarbeit. In den Szenarien mit hoher durchdringung und hohen Preisen zeigt sich
am negativen Regelarbeitsmarkt ein Trend möglichst gut die Preisspitzen mitnehmen zu können, dafür wird auf Profit am Regelleistungsmarkt
verzichtet.




Besonders im Bereich negativer Regelarbeit  ist zu erkennen, dass die Bereitstellung in Szenarien mit niedrigem und mittlerem Preisniveau
tendenziell früher erfolgt.  Dies lässt auf eine stärkere Bindung an Verpflichtungen und eine regelmäßigere Ladeplanung schließen.
In höheren Preisszenarien hingegen liegt der Fokus stärker auf einer optimalen Ausnutzung der Preisspitzen - ein Verhalten, das unter realen Marktbedingungen
nicht zwingend replizierbar ist, da es perfektes Wissen über zukünftige Preispfade voraussetzt.

Die Analyse positiver Regelarbeit verdeutlicht hingegen, dass sich Gebotsmengen in niedrig- und mittelfrequentierten Szenarien
nur geringfügig unterscheiden. Erst bei hoher Einspeisung volatiler Energien treten deutliche Unterschiede auf. Auch hier zeigt
sich: Je stärker die Restriktionen durch die Bezuschlagung im Regelleistungsmarkt, desto früher erfolgt die Bereitstellung der Regelarbeit.

Insgesamt zeigen die Ergebnisse, dass sowohl die Erwartungshaltung der Marktakteure als auch deren Strategien im Kapazitäts-
und Regelarbeitsmarkt wesentlich zur Preisbildung und zur Systemstabilität beitragen. Eine vertiefte Betrachtung
dieser Wechselwirkungen ist daher auch für regulatorische Überlegungen zur Ausgestaltung künftiger Strommärkte von zentraler Bedeutung.

\begin{enumerate}
	\item nur ein tag, eventuell kommt der richtige reload erst in zusammenhang mit mehreren tagen zum tragen
\end{enumerate}
