Diese Arbeit befasst sich mit der techno-ökonomischen Bewertung eines großskaligen Batteriespeichersystems
in Kombination mit einem Windpark, mit Fokus auf die Vermarktung am deutschen Sekundär-Regelleistungsmarkt.
Während der erzeugte Windstrom am Day-Ahead-Markt angeboten wird, nimmt der Batteriespeicher über Kapazitäts-
und Arbeitsgebote am sekundären Regelenergiemarkt teil. Grundlage der Analyse sind umfangreiche Zeitreihen, die mithilfe
statistischer Methoden erstellt und in ein effizient gestaltetes Optimierungsmodell in GAMS integriert wurden.
Die Ergebnisse zeigen, dass insbesondere in Szenarien mit hoher Einspeisevolatilität, durch erneuerbare Energien,
deutliche Preisveränderungen bei negativer Regelarbeit auftreten.\todo{den part mit den preisen drinne lassen?}
Diese Entwicklungen führen zu signifikanten Anpassungen in der Gebotsstrategie für negative Kapazitäts- und Arbeitsleistungen.