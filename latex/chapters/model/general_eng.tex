\textbf{What is represented in the model?} \\

The objective of the model is to optimize the marketing of a battery storage system in combination with a wind farm in the simplest way possible.
There are various approaches to modeling this. The battery storage system is marketed on the secondary capacity \& energy balancing markets,
while the wind farm is offered on the Day-Ahead market.
The challenge is to connect all three markets without introducing excessive complexity that would limit computational feasibility.
In addition, physical characteristics must be taken into account.
In particular, both the battery system and the grid connection point
must be accurately represented with respect to their key technical properties.
\todo{sollte hier noch eine GAMS erwähnung mit rein?}
The following section outlines the overall modeling approach.
First, the general structure and logic of the model are presented.
Subsequently, the individual market models are described in detail,
highlighting the specific regulations and formulations applied to each market.

Subsequently, the individual structural components are examined in greater detail,
with a focus on the specific constraints and modifications they introduce (see Section~\ref{chap:marketDesignDescription}).
\todo{check if constraints are in already included in this part}

Finally, the integration of these subparts into a unified optimization framework
is explained, including the mechanisms used to ensure consistency
across the different market layers (see Section~\ref{subsec:completeModel}).
\todo{end: check if order ist still correct}

This is followed by a description of how the relevant time series data
were generated for each of the considered electricity markets (see Section~\ref{chap:dataDescription}).

Finally, the chapter concludes with a discussion of the key simplifications
implemented to ensure computational feasibility of the model (see Section~\ref{chap:Simplifications}).

