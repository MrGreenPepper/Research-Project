\section{Market Design Descriptions}
\label{chap:marketDesignDescription}

The model is capable of submitting bids on three markets. Each bid consists of an amount and a corresponding price.
The bidding process starts on the secondary balance market with capacity bid, followed by the Day-Ahead market,
and finally on the secondary balancing market again with an energy bid.

The goal is to maximize the overall profit, which is composed of the amount and the price.
The amount represents the quantity offered on the market, and the price is the price at which the quantity is offered.

The resulting revenue is then given by the following equation:

$Revenue = Quantity * price$

In combination with our stochastic approach, a probability ($\omega$) is added.
This probability is always interpreted in relation to the associated market price.
While the underlying concept remains consistent, the specific interpretation of these probabilities
varies slightly across the different markets and is discussed in detail in the respective sections.
The expected revenue is calculated as the sum of all possibilities:

\begin{equation}
	Revenue = \sum_{price} Quantity(price) * price * \omega(price)
\end{equation}

The different prices are represented as different scenarios. Quantity bids can be submitted separately for each scenario,
but the total sum of the quantity bids across all scenarios is constrained.
An example of scenarios and their associated probabilities can be found in Table \ref{tab:example_scenario}.
In this case, the probability would indicate the likelihood that the bid at the corresponding price will be accepted.
This approach allows us to independently determine which quantities are allocated to each scenario.\\

\begin{table}[H]
	\centering
	\begin{tabular}{c|c|c}

		\textbf{Scenario} $s$ & \textbf{Price $p$} & \textbf{Probability $\omega(s)$} \\
		\hline
		s1                    & 90                 & 0.6                              \\
		s2                    & 100                & 0.5                              \\
		s3                    & 110                & 0.4                              \\
	\end{tabular}\\
	\caption{Example Scenario Data Table}
	\label{tab:example_scenario}
\end{table}

The objective function to be optimized for this example is as follows:

\begin{alignat}{3}
	\max Profit = & \sum_{s} Q(s) * p(s) * \omega(s) \\
	0 \leq        & \sum_{s} Q(s) \leq q_{max}
\end{alignat}

In the following chapters, the individual markets will first be described separately from \autoref{subsec:RL} to \autoref{subsec:RA}.
Subsequently, the transition from individual market problems to an overall decision will be explained [\autoref{subsec:completeModel}].

\subsection{aFFR - Capacity}
\label{subsec:RL}

\subsubsection{General Description}
The aFFr market in Germany is divided into two parts: the capacity and the energy market.
On the capacity market, bids for the provision of positive or negative balancing capacity are made for a 4-hour time window on the following day.
The auction closes at 9 a.m. on the previous day.
The settlement is made in [(Euro/MW)/h], and the paid price corresponds to the submitted bid price ("Pay-as-bid" method) [\cite{.04.12.2024}].
Whether our bid is accepted depends directly on the bid price we submit.
Since we can determine the price at which we offer capacity on the market,
we have a direct influence on the likelihood of our bid being accepted.
As a result, we can actively shape the probabilities of the respective scenarios—namely,
whether a bid is accepted or rejected.

In the case of a successful bid for capacity, bids must also be placed on the Balancing Energy Market for the same period.
The minimum bid quantity is 1 MW, and pre-qualification is required to participate.

\subsubsection{Model Implementation}
For the Frequency Control Market, the following objective function arises:

\begin{equation}
	\max Profit_{RL} = Q_{RL} * p_{RL} * \omega_{RL}(p_{RL}) \quad \forall t_{block}
\end{equation}

By transforming this into a scenario-dependent problem, the following equation results:

\begin{equation}
	\max Profit_{RL} = \sum_{t_{block}, s_{RL}} Q_{RL}(t_{block}, s_{RL}) * p_{RL}(t_{block}, s_{RL}) * \omega_{RL}(t_{block}, s_{RL})
\end{equation}


It should be noted that the quantity is now also scenario-dependent, meaning bids could theoretically be made separately for each assumed scenario.
In practice, this is not assumed, as the optimization algorithm will always assign the quantity on the most promising price-acceptance-probability combination.
Thus, the quantity serves as an abstract binary activation variable for the different price scenarios.

It should also be noted that both positive (out) and negative (in) capacity bids can be placed.
The distinction of accepted and rejected bids is determined by the probabilities $\omega$ and $1-\omega$.
While this step is not strictly necessary at this point, it simplifies the later integration of the other markets.

\begin{alignat}{3}
	\label{eq:RL}
	\\max Profit_{RL}  = 						\notag                                                                                                                        \\
	\textbf{accepted  RL in \& out:}            \notag                                                                                                      \\
	 & + \sum_{s^{out}_{RL}} \sum_{s^{in}_{RL}} (\omega^{in}_{RL}(t_{block}, s^{in}_{RL}) * \omega^{out}_{RL}(t_{block}, s^{out}_{RL}))      * (				\notag  \\
	 & + (Q^{in}_{RL}(t_{block}, s^{in}_{RL})        * p^{in}_{RL}(t_{block}, s^{in}_{RL})))				\notag                                                      \\
	 & + (Q^{out}_{RL}(t_{block}, s^{out}_{RL})      * p^{out}_{RL}(t_{block}, s^{out}_{RL}))				\notag                                                     \\
	\textbf{accepted RL in \& declined out:}        \notag                                                                                                  \\
	 & + \sum_{s^{out}_{RL}} \sum_{s^{in}_{RL}} (\omega^{in}_{RL}(t_{block}, s^{in}_{RL}) * (1-\omega^{out}_{RL}(t_{block}, s^{out}_{RL})))   * (				\notag \\
	 & + (Q^{in}_{RL}(t_{block}, s^{in}_{RL})        * p^{in}_{RL}(t_{block}, s^{in}_{RL})))				\notag                                                      \\
	\textbf{declined RL in \& accepted out:	}	\notag                                                                                                        \\
	 & + \sum_{s^{out}_{RL}} \sum_{s^{in}_{RL}} ((1-\omega^{in}_{RL}(t_{block}, s^{in}_{RL})) * \omega^{out}_{RL}(t_{block}, s^{out}_{RL}))   * (				\notag \\
	 & + (Q^{out}_{RL}(t_{block}, s^{out}_{RL})      * p^{out}_{RL}(t_{block}, s^{out}_{RL}))			\notag                                                      \\                                                                                                                                          \\
	 & \quad\forall t_{hour} = \left\lfloor \frac{t_{quarter}}{4} \right\rfloor, t_{block} = \left\lfloor \frac{t_{quarter}}{16} \right\rfloor    \notag
\end{alignat}

The constraints \ref{eq:positeQ} ensure that the offered quantities are not negative.
Additionally, it is ensured that the connection capacity $a$ is not exceeded (equation~\ref{eq:RLaccessPoint}),
and that the battery is capable of handling the corresponding power demand (equation~\ref{eq:RLrate}).

It is also important that the battery storage state can fulfill the offered power within the respective
time window \ref{eq:RL_battery_res1} \& \ref{eq:RL_battery_res2}. Note that the offered capacity is specified per hour,
while the battery storage is calculated in 15-minute intervals. Therefore, the offered capacity must be multiplied by 0.25
to obtain the value for the quarter-hour interval.

For example, if 100 MW are offered in the secondary balancing market, then for each quarter-hour in the corresponding block,
25 MWh of positive or negative work must be reserved.

\begin{flalign}
	0 \leq                                                     & Q^{in}_{RL}(t_{block}, s^{in}_{RL}),Q^{out}_{RL}(t_{block}, s^{out}_{RL})\quad\forall  s^{in}_{RL},s^{out}_{RL} \label{eq:positeQ}               \\
	r \geq                                                     & \sum_{s^{in}_{RL}}Q^{in}_{RL}(t_{block}, s^{in}_{RL}),\sum_{s^{out}_{RL}}Q^{out}_{RL}(t_{block}, s^{out}_{RL})  \label{eq:RLrate}                \\
	a + \sum_{s^{in}_{RL}} Q^{in}_{RL}(t_{block}, s^{in}_{RL}) & \geq \sum_{s^{out}_{RL}}Q^{out}_{RL}(t_{block}, s^{out}_{RL}) \label{eq:RLaccessPoint}                                                           \\
	Q^{in}_{RL}(t_{block}, s^{in}_{RL}))	* 0.25                & \leq BatCap - BatStat(t_{quarter, s_{RA}}) \quad\forall t_{block} = \left\lfloor \frac{t_{quarter}}{16} \right\rfloor \label{eq:RL_battery_res1} \\
	Q^{out}_{RL}(t_{block}, s^{out}_{RL})	* 0.25               & \leq BatStat(t_{quarter, s_{RA}}) \quad\forall t_{block} = \left\lfloor \frac{t_{quarter}}{16} \right\rfloor \label{eq:RL_battery_res2}
\end{flalign}


\subsection{Day Ahead Market}
\subsubsection{General Description}
The wind park is marketed in the Day-Ahead Market (DA). Here, bids are made for 1-hour intervals on the following day.
The auction closes at 12:00 PM the day before. The minimum bid quantity is 0.1 MWh, and bids between -500 Euros and 3000 Euros are accepted. The settlement is in [Euro/MWh],
and the price is determined using the "Pay-as-cleared" method. This means that all participants receive the price of the highest accepted bid.

\subsubsection{Model Implementation}
In parallel with the previous chapter, the equations for the Day-Ahead Market (DA) are derived.
The Day-Ahead Market is the platform where the electricity from the wind farm is marketed. Consequently, there are no positive and negative bids.
As the operator of the wind farm, we have operating costs that are close to zero, allowing us to offer electricity at a very low price.
In practice, this means that we are almost free to decide whether or not to sell electricity on the day-ahead market and receive the clearing price.
The probability $\omega_{DA}(t_{hour}, s_{DA})$ indicates the probability for the corresponding price $p(t_{hour}, s_{DA})$.
The expected profit is thus calculated as follows:\\

\begin{alignat}{3}
	\max_{Q_{DA}(t_{hour}, s_{DA})} Profit_{DA}	=            & \sum_{t_{hour}} Q_{DA}(t_{hour}) * \sum_{t_{hour}, s_{DA}}  p(t_{hour}, s_{DA}) * \omega_{DA}(t_{hour}, s_{DA}) \\
	\rightarrow max_{Q_{DA}(t_{hour}, s_{DA})} Profit_{DA}	= & \sum_{t_{hour}} Q_{DA}(t_{hour}) * p^{exp}_{DA}(t_{hour})
\end{alignat}

It is important to note that the amount of electricity generated by the wind park cannot be freely chosen.
Rather, it is constrained by the prevailing weather conditions (equation~\ref{eq:DA_capPark}).
In addition, we have the option to store the generated electricity instead of feeding it into the grid,
thus allowing the battery to be recharged (equation~\ref{eq:DA_capPark}).


\begin{alignat}{3}
	0                & \leq Q_{DA}(t_{hour}, s_{DA}) \quad\forall  t_{hour}, s_{DA}      \label{eq:DA_nonNeg}                  \\
	Q_{DA}(t_{hour}) & \leq capPark * windProfile(t_{hour}) - Q_{reload}(t_{hour}) \quad\forall t_{hour} \label{eq:DA_capPark} \\
	Q_{DA}(t_{hour}) & \leq a \quad\forall t_{hour} \label{eq:DA_a}
\end{alignat}

\subsection{aFFR - Energy}
\label{subsec:RA}
\subsubsection{General Description}
In the secondary reserve energy market, bids are placed for 15-minute intervals.
The auction closes 25 minutes before the start of the associated 15-minute block. Each prequalified
participant is allowed to bid on this market, regardless of whether a bid has been successful
in the aFFR capacity market. If a bid has been accepted in the aFFR capacity market,
a bid must also be made for the corresponding time window in the secondary market.
Payment is made only for the actual energy delivered. The dispatch of energy is based on the merit-order list,
from the cheapest to the most expensive supplier. With a high offered energy price, the likelihood of the offered
energy being called decreases. This is a "Pay-as-cleared" market, meaning all participants receive the price of the
last accepted bid. Since Germany's entry into the PICASSO network, the marginal price corresponds to the cross-border marginal price
(CBMP) \cite{50hertzamprionTENNETTRANSNETBW.}.

\subsubsection{Model Implementation}
In parallel to the aFFR capacity market, the aFFR energy market is constructed (equation~\ref{eq:RA_base}).
Since we cannot precisely predict the state of the electricity market on the following day,
multiple possible scenarios are considered, all of which are assumed to be equally likely
(for more details, see Chapter \ref{chap:affR_energy}). This applies to both positive (out) and negative (in) balancing energy.

\begin{alignat}{3}
	max Profit =  \sum_{t_{quarter}} & \biggl[ \sum_{s_{RA}}1/|s_{RA}| * p^{in}_{RA}(t_{quarter}, s_{RA}) * Q^{in}_{RA}(t_{quarter}, s_{RA})				\notag              \\
	                                 & + \sum_{s_{RA}}1/|s_{RA}| * p^{out}_{RA}(t_{quarter}, s_{RA}) * Q^{out}_{RA}(t_{quarter}, s_{RA}) \biggr] \label{eq:RA_base}
\end{alignat}

The energy is also subject to several constraints. Specifically,
the battery storage must be able to deliver the required work (equation~\ref{eq:RA_Q_out_r} - \ref{eq:RA_Q_in_Bat}),
and the connection point must have sufficient capacity (equation~\ref{eq:RA_Q_out_a} \& \ref{eq:RA_Q_in_a}).

\begin{alignat}{3}
	\sum_{s_{RA}} Q^{out}_{RA}(t_{quarter}, s_{RA}) & \leq r/4 \quad \forall s_{RA}, t_{quarter} \label{eq:RA_Q_out_r}                                    \\
	\sum_{s_{RA}} Q^{in}_{RA}(t_{quarter}, s_{RA})  & \leq r/4 \quad \forall s_{RA}, t_{quarter} \label{eq:RA_Q_in_r}                                     \\
	\sum_{s_{RA}} Q^{out}_{RA}(t_{quarter}, s_{RA}) & \leq BatStat(t_{quarter, s_{RA}}) \quad \forall s_{RA}, t_{quarter} \label{eq:RA_Q_out_Bat}         \\
	\sum_{s_{RA}} Q^{in}_{RA}(t_{quarter}, s_{RA})  & \leq BatCap - BatStat(t_{quarter, s_{RA}}) \quad \forall s_{RA}, t_{quarter} \label{eq:RA_Q_in_Bat} \\
	\sum_{s_{RA}} Q^{out}_{RA}(t_{quarter}, s_{RA}) & \leq a/4 \quad \forall s_{RA}, t_{quarter} \label{eq:RA_Q_out_a}                                    \\
	\sum_{s_{RA}} Q^{in}_{RA}(t_{quarter}, s_{RA})  & \leq a/4 \quad \forall s_{RA}, t_{quarter} \label{eq:RA_Q_in_a}
\end{alignat}

