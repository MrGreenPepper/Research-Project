\section{Simplification}

wir betrachten uns als teil eines bieterbundes
Um den rechenaufwand und die komplexität des models zu begrenzen wurden ein paar vereinfachungen vorgenommen.

Diese vereinfachungen beschränken kaum die realitätsnähe des Models.

Nochfolgende eine geordnete Aufführung welche Vereinfachungen getroffen wurden.
\subsubsection{RL}
Formal gibt es Mindestgebotsmengen jedoch können diese auch als Bieterbund erbracht werden.
Zur vereinfachung der Nebenbedingungen betrachten wir uns als teil eines bieterbundes.

\subsubsection{DA}
Da der Day Ahead markt ein pay-as-cleared markt ist haben wir mit unserem Gebot nur einen einfluss darauf ob wir aktzeptiert werden oder nicht.
Wir haben keinen einfluss darauf welcher preis für unseren strom bezahlt wird.
Da wir uns als betreiber eines Solar oder windparks betrachten haben wir betriebskosten nahe 0 die als gleich 0 angenommen werden.
Da die Day ahead preise über 0 liegen können wir als resultat in der praxis selbst entscheiden ob wir ein gebot abgeben zu einem preis der sicher aktzeptiert wird.
Daraus folgt die vereinfachte optimierung für den DA-markt aus $Profit_{Da} = Q_{DA} * erwarteterPreis_{DA}$.

\subsubsection{RA}
Da auch der RA markt ein pay-as-cleared markt ist haben wir mit unserem Gebot nur einen einfluss darauf ob wir abgerufen werden oder nicht.
Wir haben keinen einfluss darauf welcher preis für unseren strom bezahlt wird.
\todo{eventuell nochmal umformulieren da sehr ähnlich zu DA}
Da wir uns als betreiber eines Batteriespeichers betriebskosten nahe 0 haben, werden diese in folge als 0 angenommen.
So können wir den erwarteten RA markt preis soweit unterbieten das davon aus zu gehen
ist das unser Gebot auch abgerufen wird. Das gilt auch anders herum. Da wir bei einem bezugschlagtem RL Gebot
verpflichtet sind auch ein entsprechendes RA Gebot ab zu geben. So würde sich eine Nebenbedingung ergeben die die
mindest Gebotsmenge am RA markt durch die aktzeptierte gebotsmenge am RL markt begrenzt. In zusammenhang mit den eingeführten betrachten
Quantils-Szenarien würde dies aber zur einführung weiterer Dimensionen führen müssen die die rechenkomplexität unnötig erhöhen.
Praktisch lässt sich diese Regulatorische Bedingung durch einen sehr hohen Arbeitspreis umgehen der quasi sicherstellt das wir nicht abgerufen werden.
Außerdem Stellen wir sicher das für jeden zeitpunkt im bezugschlagtem RL block genügend Speicherkapazität vorhanden wäre um notfalls
den Abruf bedienen zu können. So vermeiden wir die rechenaufwendige direkt Verknüpfung von RL und RA markt und bilden trotzdem die realen mechanismen 1 zu 1 ab.

\subsubsection{Battery Storage Status}
