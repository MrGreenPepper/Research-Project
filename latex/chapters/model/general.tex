
\section{General model explanation}

\textbf{was wird in dem model überhaupt dargestellt}
\todo{noch mit rein erste entscheidung RL --> dann 4 mögliche Ausgänge!!!!!!!!!}
- profit maximierender ansatz\\
- reihenfolge der Entscheidungen\\
- wann klärt sich welche szenario unsicherheit auf\\

Ziel des Modells ist es auf möglichst einfache Weise eine Vermearktungsoptimierugn eines batterie speichers in kombination mit einem windpark vor zu nehmen.
Generell gibt es verschiedenste Möglichkeiten dies zu modellieren. Der Batteriespeicher wird am sekundären Regelleistungs und Regelarbeitsmarkt vermarktet.
Der Windpark wird am Day-Ahead-Markt angeboten. Wichtig hierbei ist es alle 3 Märkte miteinander zu verbinden ohne eine zu hohe komplexität zu benötigen die die Berechenbarkeit einschränkt.
Besonders wichtig ist dies zum Beispiel beim Batteriespeicher. Der aktuelle Ladestatus viertelstündlich neu berechnet. Selbst bei nur 2 möglichen Szenarien
wären das $ 2^{96} = 79228162514264337593543950336$ mögliche Batteriespeicher Zustände am Ende des Tages. Wenn man beachtet das die Planung immer für den Folgetag erfolgt
müsste man sogar $2^{182} = 6,13* 10^{54}$ mögliche Batteriespeicher Zustände beachten bevor man wieder Planungssicherheit hat.
\todo{den part nochmal nachdenken}
Da dies offensichtlich nicht mehr berechenbar ist muss man von perfekter Vorraussicht ausgehen und so nur einen Batteriespeicherweg berechnen.
Oder Bestimmte vorgänge innerhalb der Zeitkurve approimieren.
\todo{den part mit den speicherzuständen eventuell in die Modelierungsansatz Diskussion}

Lösungsansätze für dieses und andere Probleme sind im Abschnitt [] zu finden
\todo{abshcnitt eränzen}
Weiterhin ist zu beachten des sich der Windpark und der Batteriespeicher einen gemeinsamen Anschlusspunkt teilen, so ist die maximale Leistung beider begrenzt.

Es folgt eine Diskussion verschiedener Modellansätze. Anschließend werden die Einzelmodelle der verschiedenen Märkte betrachtet und zum schluss zusammen geführt.\\
-------------------------------------\\
\todo{aussortieren was noch mit oben rein soll}
Zur Analyse des vorliegenden Problems wurde ein Model in GAMS erstellt.
Ziel des Models war es auf möglichst geringem Rechenaufwand einen Batteriespeicher zu optimieren der mit
einer Analage zur produktion erneuerbarer energien kombiniert wurde. Dabei sollte vermmieden werden
auf sehr detailierte Zeitreihenvorhersagen, weil sehr aufwendig, angewiesen zu sein. Es sollten aber auch
Grundannahmen wie perfekte Vorraussicht vermieden werden um realistische Planungsentscheidungen ab zu bilden.

(Zur Vereinfachung werden zuerst alle Formeln für nur einen Zeitschritt aufgestellt. Am Ende wird die Zeitvariable entsprechend hinzugefügt.)
\todo{ausführliche Erklärung stochastische Programmierung}
\\
Das grundlegende Modell stellt einen Energiespeicher da, der am Regelleistungsmarkt, Day Ahead Markt und Regelarbeitsmarkt vermarktet werden kann.
Der darraus resultierende Profit soll maximiert werden. Für jede Teilentscheidung/Markt existiert ein eigenes Modell. So kann, für jedes Teilmodell,
vermieden werden die anschließenden Marktergebnisse (Zuschlag oder Ablehnung) zu integrieren. Dies ist wichtig, da anderen Falls der
Algorithmus allwissend wäre und nur perfekte Gebote errechnen würde. Die Ergebnisse eines jeden Teilmodells werden immer an das nachfolgende
Modell übergeben und erst an dieser Stelle ausgewertet. Jedes Teilmodell ermittelt Mengengebote zu bestimmten Preisen. Die verschiedenen Preise
werden durch verschiedene Szenarien abgebildet. Jedem Szenario ist eine bestimmte Wahrscheinlichkeit zugeordnet.
(Die Preis-Wahrscheinlichkeits-Kombinationen der verschiedenen Szenarien wurden vorher exogen mittels SARIMA-Analyse ermittelt.)
Ein Gebot stellt sich dann wie folgt dar:\\
---------------------------------\\

- modeldesign erklären
-> verschiedene designoptionen dann "design optionen"/alternativen/erklärung?!? part erklären
normalerweise liegt die logik in den daten und ich lasse den solver die logik in den daten erkennen.
wenn ich aber keinen realistischen vorcast daten habe muss ich die logik in das programm schon selber legen.

das wesentliche meiner

- anschluss punkt

- kombination aus park und batterie

\section{Market Modeling Approaches}
\textbf{ansatz wie das darzustellende umgesetzt wird damit es optimierbar ist} \\
- eventuell in Konzepte umbenennen und ganz allgemein über verschiedene konzepte sprechen\\

Verschiedene Modellierungsansätze erfordern unterschiedliche zu grunde liegende Datensätze und anders herum.
So erfordert zum Beispiel ein stochastische model, welches eine optimierung über mehrere unsichere mögliche Szenarien vornimmt,
einen Datensatz der diese verschiedenen Szenarien abbildet. Bei der direkten verwendung historischer Daten benötigt man ein model,
welches  von einer perfekten Vorhersage ausgeht und so nur eine Datenreihe berücksichtigt.

Im folgenden werden verschiedene Ansätze bezüglich Model und Daten besprochen.
\todo{gewinnmaximierung}
%\subsubsection{Modelansätze Marktdesigns}

Ein möglicher Modelierungsansatz ist der der dem Model die perfekte Vorhersage unterstellt, hier werden historische
Daten eingespeist und anschließend die Ergebnisse
unter dieser Prämisse betrachtet. So werden dann die perfekten Ergebnisse mit einem abgeschätzen Prozentsatz herunterskaliert um zu einem
realistisch erzielbaren ergebnis zu kommen. Der Vorteil dieser Methode in unserem Fall läge in der einfachen Komplexität. Da man immer
genau weiß was eintritt muss auch nur ein einzelner Zeitstrahl verfolgt werden.
Der Nachteil liegt ganz klar darin das man bei der Betrachtung der Ergebnisse eventuell auf falsche Strategien schließt.
So müssen in  unserem Fall Entscheidungen getroffen werden mit unsicherer Zukunft Szenarien.
So kann es sein das zum Beispiel die Anschlusskapazität es nicht zulässt zugleich am DA-Markt und am RA- zu bieten.
\todo{abkürzungen} Bei perfekter Vorraussicht weiß ich genau welche entscheidung bei gegeben Daten die richtige ist auch wenn
der Unterschied zwischen den beiden Entscheidungen nur marginal ist.
Dies ist dann aber nur eine Einzelfallentscheidung, in realität kann es aber sein, dass eine andere strategie, über mehrere Fälle hinweg sich
als vorteilhaft herausstellt.
\todo{verweis wissenschaftliche arbeit und appendix für umsetzung}
So lassen sich mit diesem Ansatz gut Einzelfall entscheidungen treffen aber nicht gut auf eine allgemeine Strategie schließen.
Um eine Allgemeinere Strategie ableiten zu können bieten sich stochastische programmieransätze an. Diese bestimmen optimale Entscheidungen
unter betrachtung  mehrerer möglicher unsicherer Zukunftszenarien. So wird im Model eine entscheidungsvariable mit mehreren szenarien und deren
eintrittswahscheinlichkeiten kombiniert um so auf eine best mögliche Entscheidung unter Unsicherheit zu schließen.
So lassen sich einfacher optimalere allgemein gültige Strategien ableiten, allerdings ist hier die Erstellung der dafür notwendigen Daten wesentlich
schwieriger. So braucht man verschiedene Datensätze die die verschiedenen Szenarien präsentieren und muss diese Datensätze auch mit eintrittswahrscheinlichkeiten bewerten.
Im folgenden werden mehrere Ansätze diskutiert wie man dies für Zeitreihen-Datensätze macht \todo{abschnitt ergänzen}.
Außerdem müssen müssen oft vereinfachende Annahmen getroffen werden um die Modelkomplexität zu reduzieren.
Besonders wichtig ist dies zum Beispiel beim Batteriespeicher. In unserem Model wird der aktuelle Ladestatus des Batteriespeichers viertelstündlich neu berechnet. Selbst bei nur 2 möglichen betrachteten Szenarien
müsste man $ 2^{96} = 79228162514264337593543950336$ mögliche Batteriespeicher Zustände am Ende des Tages betrachten. Wenn man beachtet das die Planung immer für den Folgetag erfolgt
müsste man sogar $2^{182} = 6,13* 10^{54}$ mögliche Batteriespeicher Zustände beachten bevor man wieder Planungssicherheit hat. Deswegen werden erfolgt eine Betrachtung verschiedener Vereinfachungen in Kapitel []
\todo{den part nochmal nachdenken}\\
\todo{abschnitt ergänzen}\\
In Summe sind die Vorteile eine stochastischem Ansätzes größer, vor allem um die hier vorliegende Forschungsfrage nach allgemein güiltigen Strategien
zu beantworten.
Zur erstellung von optimalen allgemeinen strategien \\



