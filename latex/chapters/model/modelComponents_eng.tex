
\section{Model components}
The previous presented equations served to calculate the profit to be maximized.
BUt, there are fundamental real-world components that must be reflected in our model,
which impose constraints on the profit-maximizing formulations.

We begin by examining the battery system and its charging capabilities in more detail.
Subsequently, the modeling of the grid connection point is discussed.

\subsection{Battery \& Charging Capabilities}
The fundamental properties of the battery storage are described through parameters combined with constraints.
The battery storage has a maximum charging and discharging reate $r$ and a maximum capacity $BatCap$.
The battery status $BatStat$ is recalculated every quarter-hour and can be found in equation~\ref{eq:BatStat}.
Since the recharge amount $Q_{reload}$, which comes from the wind park, is calculated hourly,
it must be converted to quarter-hourly values. The battery status for the timestep $t_{quarter} + 1$
is derived from the battery status at the previous timestep $t_{quarter}$, along with the actual negative reserve energy delivered,
minus the actual positive reserve energy delivered. Furthermore, there is the option for a working point adjustment $WP$.
This adjustment can be made to align the battery's charge level with the obligations that need to be met.

\begin{alignat}{3}
	BatStat(t_{quarter} + 1)  = BatStat(t_{quarter}) + & \frac{1}{4} Q_{reload}(t_{hour})\notag                                                               \\
	                                                   & +\sum_{s_{RA}}1/|s_{RA}|  * Q^{in}_{RA}(t_{quarter}, s_{RA})\notag                                   \\
	                                                   & - \sum_{s_{RA}}1/|s_{RA}|  * Q^{out}_{RA}(t_{quarter}, s_{RA})\notag                                 \\
	                                                   & - \sum_{s_{RA}}1/|s_{RA}|  * WP_{out}(t_{quarter}, s_{RA})\notag                                     \\
	                                                   & + \sum_{s_{RA}}1/|s_{RA}|  * WP_{in}(t_{quarter}, s_{RA})\notag                                      \\
	                                                   & \forall t_{quarter}, t_{hour} = \left\lfloor \frac{t_{quarter}}{4} \right\rfloor  \label{eq:BatStat} \\
	0 \leq  BatStat                                    & (t_{quarter}) 			\label{eq:BatStat_nonNeg}                                                           \\
	BatStat(t_{quarter}) \leq  BatCap                  & \label{eq:BatStat_cap}
\end{alignat}


\subsection{Grid Connection}
The wind farm and the battery storage system
share a common grid connection. As a result, their combined output is subject to
a maximum power constraint, which must be respected throughout the optimization process.
The maximum power that can flow through the connection point is $a$. This power limit applies in both directions,
as shown in equations \ref{eq:a_general_pos} and \ref{eq:a_general_neg}. Since this constraint applies to all quarter-hour
intervals, we have to divide the work of the wind farm and the grid connection maximum power by four.

\begin{alignat}{3}
	\frac{a}{4} + Q^{in}_{RA}(t_{quarter}, s_{RA}) +  WP_{in}(t_{quarter}, s_{RA}) \notag                                   \\
	\geq \frac{1}{4} Q_{DA}  + Q^{out}_{RA}(t_{quarter}, s_{RA}) + WP_{out}(t_{quarter}, s_{RA}) \notag                     \\
	\quad \forall s_{RA},  t_{quarter},  t_{hour} = \left\lfloor \frac{t_{quarter}}{4}\right\rfloor\label{eq:a_general_pos} \\
	\frac{a}{4} + \frac{1}{4}Q_{DA} +  Q^{out}_{RA}(t_{quarter}, s_{RA}) + WP_{out}(t_{quarter}, s_{RA}) \notag             \\
	\geq Q^{in}_{RA}(t_{quarter}, s_{RA}) +  WP_{in}(t_{quarter}, s_{RA}) \notag                                            \\
	\quad \forall s_{RA},  t_{quarter},  t_{hour} = \left\lfloor \frac{t_{quarter}}{4}\right\rfloor \label{eq:a_general_neg}
\end{alignat}

\todo{strikte variante}
