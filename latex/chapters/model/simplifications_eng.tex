\section{Simplifications}
\label{chap:Simplifications}
In this study, we assume the role of a participant within a bidding consortium.
As such, minimum bid requirements can be disregarded,
as we assume our consortium partners to be sufficiently large to always meet these minimum thresholds.

To manage the computational effort and reduce the model's complexity, several simplifications have been implemented.
These simplifications do not substantially compromise the realism of the model.
The following is a structured list of the simplifications employed.

\subsubsection{aFFR Capacity/DA Quantile Data}
For the time series analyzed, it is important to note that the values are determined based on the preceding day.
As a result, the prediction error for the following day has not yet materialized.
Consequently, the quantile data for the day ahead aFFR capacity can be averaged over the 10 scenarios days.
This approach minimizes unnecessary complexity and allows for more general,
strategic conclusions to be drawn regarding bidding behavior in the aFFR capacity/DA market,
in relation to potential high, medium, and low prediction errors.

\subsubsection{Day Ahead Market}
Since the Day-Ahead market operates under a pay-as-cleared structure, our bid influences only the acceptance or rejection of our offer,
with no impact on the price paid for the electricity.
Given our assumption that we operate a solar or wind farm, we consider operational costs to be negligible, and thus, assume them to be zero.
As Day-Ahead prices are non-negative, we can effectively choose whether to submit a bid at a price that will almost certainly be accepted.
This results in a simplified optimization for the DA market, represented by $Profit_{Da} = Q_{DA} * p^{exp}_{DA}$.

\subsubsection{aFFR Energy Market}
As the aFFR Energy Market market also follows a pay-as-cleared principle, our bid influences only whether we are called upon to deliver, with no impact on the price paid for the electricity.
Assuming the role of a battery storage operator with near-zero operational costs, these are considered negligible.
Thus, we can submit a bid below the expected RA market price to ensure that our offer is likely to be called.
The reverse is also true.

When an aFFR capacity bid is accepted, we are obligated to submit a corresponding aFFR energy bid.
This creates a constraint that limits the minimum bid quantity in the aFFR energy market based on the accepted aFFR capacity bid quantity.
However, integrating the introduced case would lead to an increase in computational complexity due to the need for extra variables and additional
dimensions. In practice, this regulatory constraint can be bypassed by setting a sufficiently high working price, ensuring that our bid is not called upon.
Therefore, this constraint was not explicitly incorporated into the model.
But, we incorporated a constraint that links the minimum and maximum $BatStat$ to the accepted aFFR energy bids.
This ensures that, at all times during the relevant aFFR block, sufficient storage capacity is available to fulfill any potential requests.
This approach avoids the computationally intensive direct linkage between the aFFR capacity and aFFR energy markets while still
accurately reflecting the underlying mechanisms.

\subsubsection{Battery Storage Status}



The battery storage status is recalculated at 15-minute intervals.
The battery status from the previous time window is updated by adding all inflows and subtracting all outflows.
Given the uncertainty of these inflows and outflows, multiple possible battery states would arise at the end of each calculation window, based on the possible preceding states.
Since we calculate the battery status over 96 consecutive time periods, the computational complexity would increase exponentially due to the number of possible outcomes over these 96 periods.
Even with only two considered scenario outputs, this results in
\[
	2^{96} = 79228162514264337593543950336
\]
possible battery storage states at the end of the day.
Considering that planning always takes place for the following day,
one would need to account for
\[
	2^{182} = 6.13 \times 10^{54}
\]
possible battery storage states before achieving planning certainty again.
Since a full enumeration of all possible future system states quickly becomes computationally infeasible,
it is necessary to either assume perfect foresight and calculate a single, most probable trajectory
for the battery storage system, or to approximate certain processes along the time series.
Therefore, we approximate the battery storage calculation by determining an expected battery status at time $t_quarter$ and using this expected value for subsequent calculations.
Inflows and outflows are weighted by their respective probabilities, allowing for an approximation of the correct battery status over the entire period.

This method inherently flattens the battery loading status curve, depending on the assumed probability distributions.
For example in reality, consecutive 10\% fluctuations may occur, leading to a more pronounced shift in the battery status than the model predicts.
To ensure that the storage capacity meets the required levels, the actual capacity have to be recalculated by considering unweighted quantity bids and determining the maximum fluctuation.
This value corresponds to the actual required storage capacity.
As the storage/bid ratio increases, the impact of this adjustment becomes negligible, as the likelihood of consecutive unlikely events diminishes over time.

