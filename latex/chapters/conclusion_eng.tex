\chapter{Conclusion}


The analyses presented in this work clearly demonstrate that increasing market penetration by volatile energy producers
has significant effects on both the utilization and pricing of balancing markets. In particular,
it is observed that with a growing share of volatile generation, the volume of activated negative balancing energy increases,
while demand for positive balancing energy decreases. This pattern is also reflected in the marginal prices: while the prices
for negative balancing energy tend to rise with increasing generation volatility, the prices for positive balancing energy show
a declining trend.

Notably, strong price outliers can be observed for positive balancing energy in scenarios with low or medium market penetration
by volatile producers. These anomalies appear to be driven by unexpectedly high demand peaks in scenarios where market participants
did not anticipate major fluctuations. This suggests that the expectations and preparedness of market participants play a crucial
role in maintaining price stability under conditions of variable renewable energy feed-in.

The analyzed aFFR capacity prices reveal a nuanced picture: in median quantile scenarios, values peak for both positive and negative balancing capacity,
indicating decreasing supply in this setting.
This may be due to the fact that, in the respective aFFR energy scenarios
with attractive price levels, the corresponding capacity market
is perceived as a side benefit.
This behavior leads to a supply increase that exceeds the rise in demand, resulting in falling capacity prices.
In scenarios with low aFFR prices, the underlying cause appears
to be a generally low level of demand.
However, a conclusive analysis requires further investigation.

Across all scenarios, the bidding strategies for the balancing capacity market prices are positioned just below the expected marginal prices.

In scenarios with low to medium levels of volatile production penetration,
the introduction of constraints through the aFFR capacity market leads to a shift
towards earlier provisioning in the negative balancing energy market. This is reflected in a more regular schedule of negative balancing energy delivery. In contrast, in high volatility
scenarios with elevated prices, a strategy focused on capitalizing on price peaks becomes evident. In such cases, providers may choose to forgo profits
in the capacity market in order to place their bids more flexibly in the energy market.
This strategy aims to take better advantage of price peaks in the aFFR energy market.

It is also worth noting that, while the model accounts for different possible aFFR energy market scenarios,
it assumes perfect foresight within each individual daily scenario.
As a result, the revenue potential from the aFFR energy market is likely overestimated.
Moreover, the model heavily exploits price peaks, which may not be fully realizable in practice.
Whether such behavior can be implemented in reality depends largely on the quality of intraday forecasts
and requires further investigation.

An alternative modeling approach would be to refrain from using precise price data for the aFFR energy market
and instead attempt to represent general market cycles.
Under the hypothesis that the transmission system operator continuously strives to balance the grid,
it can be postulated that any imbalance will eventually be corrected.
Prolonged imbalances are assumed to be relatively unlikely.
Based on this assumption, certain recurring patterns could be forecasted within a 4-hour window,
characterized by a defined expected value.
Appendix \ref{app:altModel} presents a preliminary model approach for this idea.
However, both the scientific validation of the underlying hypothesis and the development of a suitable aFFR prediction cycle algorithm
remain subjects for further research.

For positive balancing energy, the analysis reveals that bidding volumes in low and medium volatility scenarios differ only marginally
depending the aFFR capacity market.
Pronounced differences emerge only in high-volatility scenarios. Again, the data show that the more restrictive the capacity market
constraints, the earlier the provision of balancing energy occurs.

Since the model considered only a single day,
it is possible that the positive effects of battery charging
via the wind farm have not yet been fully realized.
Additional benefits may arise from the battery serving as a backup in cases of forecasting errors
in renewable energy generation, thereby helping to avoid potential costs.
To better assess both effects, further research is required.

In summary, basic strategic guidelines can be derived from our model.
However, further research is required to develop precise decision criteria
and the resulting actionable strategies.
