\chapter{Introduction}

The accelerating transition towards renewable energy sources presents both opportunities and challenges for modern
power systems. But, the inherent variability and limited predictability of renewable energy generation pose
challenges to grid stability. As a result, there is a increasing demand for flexible technologies such as battery energy
storage systems (BESS) to ensure reliable grids.

In particular, the provision of ancillary services—especially frequency regulation has emerged as a promising
revenue stream for storage technologies. Germany's balancing markets, including the secondary
control reserve (aFRR), offer significant potential for batteries due to their fast ramping capabilities and
high availability.

While renewable energy sources primarily participate in the day-ahead market based on forecasted production,
battery systems mostly operate on balancing markets.

\todo{hier noch mit rein das batterie speicher in kombination mit einem windpark positive spillovers haben kann}
deploying such a BESS in conjunction with renewable energy plants, battery storage systems offer complementary capabilities.
On the on hand the BESS can be reloaded by the renewable energy, avoiding market fees, on the other hand the renewable energy production
can be time shifted towards time windows with higher energy prices. Also combining renewable energy generation with battery storage in
a co-located hybrid system allows operators to diversify revenue streams by participating simultaneously in multiple electricity markets.


\todo{- vor und nachteile, begrenzte kapazität des anschlusspunkts, überschüssige energie kann zum laden genutzt werden.}
However, such joint operation requires sophisticated optimization techniques that consider market mechanisms, physical constraints,
and operational synergies.

hier literatur teil

There has been various research on this field.

Studie batterie für sekundären Markt
Studie für Windkraft

--> ich fokus auf batterie in kombination sowie vereinfachendes model


In this context, mathematical programming tools such as GAMS (General Algebraic Modeling System)
are well-suited to model and solve complex multi-market dispatch problems.

The objective of this approach is to determine an optimal marketing strategy for battery energy storage
systems that simultaneously accounts for participation with both facilities in all three relevant electricity
markets.
In practice, this requires a sequence of three interdependent decisions: first, submitting a bid in
the balancing capacity market, followed by participation in the day-ahead market, and finally,
submitting a bid in the balancing energy market.

This paper presents a optimization model developed in GAMS to simulate the joint operation of a
wind farm and a co-located battery storage system. The wind farm's revenues are derived from the German day-ahead
electricity market, while the battery system participates in the secondary balancing market.
The model maximizes total system profit while adhering to market and technical constraints.

To achieve this objective, time series data for each individual market were simulated,
and representative scenario cases were subsequently extracted. These scenarios were then
implemented into a GAMS optimization model, which was used to compute the optimal marketing
strategy for the battery storage system.

The following chapter provides a brief overview of the current state of research. Chapter 3 offers a detailed
discussion of the applied methodology. It begins with an explanation of the general modeling approach,
followed by an in-depth examination of each modelpart. Furthermore, the process of simulating the relevant
time series and selecting representative scenarios is described.
-----------------------------------------\\


(This work contributes to the growing body of research on hybrid renewable systems by addressing the
integration of distinct market participation strategies and quantifying the economic benefits of coordinated operation.)
\todo{nochmal lesen und schauen ob roter faden gut}

