\chapter{Introduction}

The accelerating transition towards renewable energy sources presents both opportunities
and challenges for modern power systems. However, the inherent variability and limited
predictability of renewable generation pose significant threats to grid stability.
As a result, the demand for flexible technologies—such as battery energy storage systems (BESS)—is increasing,
to ensure a reliable and resilient energy supply.

In particular, the provision of ancillary services—especially frequency regulation—has emerged
as a promising revenue stream for storage technologies. Germany\textquotesingle s balancing markets,
including the secondary control reserve (aFRR), offer significant potential for battery systems,
thanks to their rapid ramping capabilities and high operational availability.

While renewable generators primarily participate in the day-ahead market based on
forecasted production, battery storage systems are typically deployed on balancing markets.
Operating a BESS in conjunction with a renewable power plant provides several technical and economic advantages.

On the one hand, excess renewable electricity can be used to charge the battery, thereby
avoiding market-related fees. On the other hand, generation can be time-shifted to periods
with higher energy prices. Moreover, the co-location of renewable generation and storage in a hybrid system
enables operators to diversify their revenue streams by participating in multiple electricity markets simultaneously.

\todo{Check whether the market participation constraint is discussed later in the model section—
	if not, consider briefly mentioning it here as a potential downside.}

However, such joint operation requires advanced optimization techniques that account for market
mechanisms, physical constraints, and operational synergies.

In this context, mathematical programming tools such as GAMS (General Algebraic Modeling System)
are well-suited to model and solve complex multi-market dispatch problems.

The objective of this study is to determine an optimal bidding strategy for a battery energy storage system
co-located with a wind farm, across three relevant electricity markets. These include the day-ahead market,
the balancing capacity market, and the balancing energy market.

In practice, this requires a sequence of interdependent decisions:
first, the submission of a capacity bid in the balancing market;
second, participation in the day-ahead energy market;
and finally, submission of an energy bid in the balancing energy market.

This paper presents an optimization model developed in GAMS to simulate the joint operation of
a wind farm and a co-located battery storage system. While the wind farm's revenue is derived
from the German day-ahead electricity market, the battery system participates in the
secondary balancing market.

The model aims to maximize total system profit while respecting both market rules
and technical constraints. To this end, synthetic time series data were generated for each market using statistical methods.
Representative scenarios were then selected and implemented into the GAMS model to compute
an optimal bidding strategy for the storage system.

The next chapter provides a brief overview of the current state of research.
Chapter 4 describes the applied methodology in detail, including the general modeling framework
and the individual components of the optimization model.
Additionally, the process of generating market time series and selecting representative scenarios
is discussed. Chapter 5 presents the results, followed by a summary and conclusion in Chapter 6.
