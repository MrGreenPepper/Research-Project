This study addresses the techno-economic evaluation of a large-scale battery storage system
in combination with a wind farm, with a particular focus on participation in the German
secondary control reserve market. While the wind-generated electricity is marketed via the Day-Ahead market,
the battery storage system participates in the secondary reserve market through
capacity and energy bids. The analysis is based on extensive time series data,
which were generated using statistical methods and integrated into
an efficiently designed optimization model implemented in GAMS.
The results indicate that, especially under scenarios characterized by high feed-in volatility
from renewable energy sources, it is beneficial to maintain negative battery capacity
for extended periods and wait for probable price peaks.
This development leads to significant adjustments
in the bidding strategy for negative capacity and energy provision.
\todo{nochmal checken ob die ergebnisse den letzten satz so hergeben}
