This study addresses the techno-economic evaluation of a large-scale battery storage system
in combination with a wind farm, with a particular focus on participation in the German
secondary control reserve market. While the wind-generated electricity is marketed via the Day-Ahead market,
the battery storage system participates in the secondary reserve market through
capacity and energy bids. The analysis is based on extensive time series data,
which were generated using statistical methods and integrated into
an efficiently designed optimization model implemented in GAMS.
The results indicate that, particularly in scenarios characterized by high feed-in volatility
from renewable energy sources and a non-constraining balancing capacity market,
it is advantageous to maintain available negative battery capacity over extended periods
in order to capitalize on potential price peaks.

