\chapter*{Vorwort}

Ich habe lange überlegt ob ich dieses Vorwort schreiben soll oder nicht, aber ich möchte doch nochmal gerne
darauf eingehen wie diese Arbeit entstanden ist und wo Schwierigkeiten lagen.

Angefangen hat das ganze mit dem Auftrag einen Speicher zu optimieren der an dem sekundären Regelleistungsmarkt angebungen ist und mit einem Wind- oder
Solarkraftwerk kombiniert ist.

Zur Optimierung an den einzelnen Märkten gibt es eine Vielzahl an arbeiten.

Die dabei benutzten optimierungsmodelle sind diverse.

Sie unterscheiden sich hinsichttlich der betrachteten Märkte als vermaktungsoption, hinsichtlich der benutzen Modelansätze, hinsichtlich der genutzten Daten und oft wird nur ein
batteriespeicher ohne kombunuerten erneurbaren energie produzent optimiert.

So betrachten viele Modellansätze nur einzelne Märkte. \todo{cite}


Die Modellansätze unterscheiden sich hauptsächlich darin ob eine perfekte Vorraussicht unterstellt wird oder nicht. \cite{Nitsch.2021} \& \cite{Olk.2019}
zum Beispiel beschreiben den den sekundären Regelleistungsmarkt mit einem solchem Model mit perfekter Vorraussicht. Kombiniert mit exakten Realwelt Daten
ergibt sich daraus ein maximal zu erreichender Ertrag für den Energiespeicher. Jedoch lassen sich aus der Kombination
von Modell mit perfekter Vorraussicht und exakten Daten schlecht allgemeine Strategien ableiten, da man auch die Ausnahmefälle
perfekt einplanen kann.

Um diese Problemstellung besser an gehen zu können werden hauptsächlich 2 methodische Ansätze verfolgt. Entweder man limitiert das Model hinsichtlich
seines Wissens über zukünftige Ereignisse und bildet so die in der realen welt inherente Unsicherheit ab. Oder man beschränkt nicht die fähigkeiten des
Modells, sondern modifziert die Daten das diese die Unsicherheit abbilden. Solche Daten wären dann zum Beispiel Vorhersage Prognosen die
die einen Erwartungswert berechnen, oder Daten die mehrere mögliche Szenarien darstellen kombiniert mit einer berechneten Eintrittswahrsceinlichkeit für
die jeweiligen Szenarien \cite{Krishnamurthy.2018}. Wärend dieser Ansatz gut ist um die Unsicherheiten ab zu bilden, kann es jedoch sehr schwer sein
die entsprechenden Daten zu ermitteln. Vor allem Daten für Märkte vorher zu sagen die sehr vielen Einflüssen unterliegen und sehr unregelmäßig sind
gestaltet sich dabei als besonders schwierig. Das gilt in unserem Fall für den sekundären Regelarbeitsmarkt. Dies ist der Markt
an dem relativ kurzfristig die tatsächlich beötigte Energie gehandelt wird um das Netz aus zu gleichen. \cite{OConnor.2024} beschäftigen sich ebenfalls mit
dem Problem der Preisvorhersage an den Regelenrergie Mörkten. Sie zeigen auf das selbst mit sehr komplexen Modellen eine es schwierig ist den Balancing Preis gut vorher zu sagen.
Für den hier vorliegenden Fall ist dies besonders kritisch, da die ersten Gebote schon am Vortag abgegeben werden müssen, zu diesem Zeitpunkt es aber sehr schwer ist vorher zu sagen
wie der genaue Netzstatus am Folgetag zum Zeitpunkt x
sein wird. Damit ist unklar welche Regelarbeit zum Zeitpunkt x benötigt wird.


Außerdem werden in all diesen Arbeiten nicht die Kombination (BASS, erneurbarer Erzeuger, aFFR, DA Markt) betrachtet.

Die Herausfordung für diese Arbeit ist es nun die einzelnen Ansätze für die einzelnen Probleme zu kombinieren.
Dabei stellen sich 2 Hauptprobleme. Zum einen müssen die gewählten Modelllösungsansätze und Daten für die einzelnen Teilmärkte technisch kombinierbar sein.
Zum anderen darf die Komplexität nicht explodieren. Wärend es noch relativ einfach ist ein komplexes Teilproblem (in unserem Fall
eine einzelne Marktbetrachtung) zu lösen. So Steigt die komplexität expotentiel um so mehr teilprobleme kombiniert werden.
Dies betrifft sowohl die einfache Umsetzung, als auch ganz direkt die schlichte Berechenbarkeit.\todo{formulierung komplexitäts explossions abschnitt}

So versucht diese Arbeit die Kombination aus Batterie und erneurbaren Erzeuger zu

... ohne die Komplexität explodieren zu lassen und trotzdem verwertbare Strategien ab zu leiten.

... auflistung der Kapitel ... innerhalb des Methodischen Abschnitts wird auch nochmal explizit auf die gewählten vereinfachungen ein gegangen, die
gewählt wurden um die Komplexität zu begrenzen.

Zum einen gibt es optimierungsmodelle die eine perfekte Vorraussicht unterstellen


Electricity price modeling with stochastic time change

ein markt: Electricity Price Forecasting in the Irish Balancing Market

eventuell einführung Regelarbeitsmarkt noch mit rein

Bidding strategy for a battery storage in the German secondary balancing power market ... aber altes system

------------------------------
cite:
nur einzelne Mörkte:


perfect foresight:
Economic Value of Energy Storage Systems: The Influence of Ownership Structures ... perfect foresight
An Optimal Energy Storage Control Strategy for Grid-connected Microgrids ... perfect demand foresight
Economic evaluation of battery storage systems bidding on day-ahead and automatic frequency restoration reserves markets ... perfect full foresight
Bidding strategy for a battery storage in the German secondary balancing power market ... perfect full foresight


