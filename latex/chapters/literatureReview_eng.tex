\chapter{Literatur Review}

There is a wide range of research focused on optimization in individual energy markets.
The optimization models used vary greatly.
They differ with respect to the markets considered as marketing options,
the modeling approaches applied, the data used, and often they focus solely on optimizing
battery storage systems or a renewable power plant.

Many models consider only individual markets.  Olk et al. \cite{Olk.2019} consider only the secondary german balancing market and
optimize the battery in this context . Cai et al. \cite{Cai.2016} propose an optimization for a battery combined with an wind park, but
focus on the revenue from the park at the day ahead market and use the battery to avoid penalties due to prediction errors.

A core challenge in developing such models lies in how to represent the uncertainty
that is inherently present in real-world operations. For example, models that assume perfect
foresight and rely on exact historical data
are capable of calculating the theoretical maximum return for the system under consideration.
But, such models fail to reflect the uncertainty and operational risks
that decision-makers face in practice, which limits the applicability of the resulting strategies.

To better address this issue, two methodological approaches are commonly used:
Either the model\textquotesingle s knowledge of future events is restricted,
or the data is modified to represent the uncertainty.
For instance Nitsch et al. \cite{Nitsch.2021} modeled an optimization problem with perfect foresight but predicted data.
Such data could include forecast values that represent expected values or scenario-based data with associated
probabilities of occurrence, as discussed by Krishnamurthy et al. \cite{Krishnamurthy.2018}.
While this approach is useful for representing uncertainty, obtaining suitable data can be very
challenging, especially for markets that are highly volatile and influenced by many factors.
This is particularly true for the secondary balancing energy market, where energy is traded at short notice to maintain grid stability.
OConnor et al. \cite{OConnor.2024} also address the difficulty of predicting prices in balancing energy markets .
They demonstrate that even with highly sophisticated models, forecasting balancing prices remains a significant challenge.
In the present case, this becomes especially critical since initial bids must be submitted the day before at a time when it is
extremely difficult to predict the exact grid status at any given moment the next day.
As a result, it is unclear what kind of balancing energy will be required at that time.

Moreover, much of the existing literature either does not reflect current market regulation
or fails to consider the joint operation of all components relevant to this work:
a battery energy storage system (BESS), a renewable generator, participation in the Day-Ahead (DA) market,
and the provision of automatic frequency restoration reserve (aFRR).

The core challenge of this thesis lies in combining existing modeling approaches
for each of these individual components into a single, cohesive optimization model.
This integration involves three major difficulties:

\begin{itemize}
	\item \textbf{Model compatibility:} The chosen modeling approaches and data structures
	      for the individual submarkets must be technically compatible with one another.

	\item \textbf{Managing complexity:} Although each subproblem can be solved independently,
	      the overall complexity increases exponentially when combining multiple markets and decision levels.
	      Therefore, simplifications are necessary to maintain computational feasibility
	      without losing essential system dynamics.

	\item \textbf{Data integration/timeline prediction:} The aforementioned points also apply to the integration
	      of time series data. Forecasting and scenario generation must be harmonized across all submarkets
	      to ensure realistic and consistent input for the optimization model.
\end{itemize}

These challenges affect not only the conceptual design of the model but also its practical implementation and runtime performance.
Accordingly, a key contribution of this work is the development of a simplified yet representative optimization framework
that integrates these components into a tractable form.
Unlike many previous studies, this work focuses on the joint optimization of multiple markets and system components,
while implementing scenario-based forecasting data to better represent operational uncertainty.\\
